%
\documentclass[12pt,notitlepage]{article}
\usepackage{amssymb}
\usepackage{amsmath}
\usepackage{graphicx}
\usepackage{epstopdf}
\usepackage{pdflscape}
\usepackage{tabularx}
\usepackage{longtable}
\usepackage{array}
\usepackage{dsfont}
\usepackage{float}
\usepackage{booktabs}
\usepackage{marvosym}
\usepackage{multirow}
\usepackage{pdflscape}
\usepackage[hyphenbreaks]{breakurl}
\usepackage[hyphens]{url}
\usepackage{setspace}
\usepackage{epigraph}
\usepackage{bm}
\usepackage{textcomp}
\usepackage{bbm}
\usepackage{verbatim}
\usepackage{subcaption}
\usepackage{caption}
\usepackage[shortlabels]{enumitem}
\usepackage{graphicx}
\setlength{\epigraphrule}{0pt}
\setlength\parindent{3em}
\renewcommand{\baselinestretch}{1.5}
\renewcommand*{\arraystretch}{1.2}

\setcounter{MaxMatrixCols}{10}

\newcolumntype{L}[1]{>{\raggedright\let\newline\\\arraybackslash\hspace{0pt}}m{#1}}
\newcolumntype{C}[1]{>{\centering\let\newline\\\arraybackslash\hspace{0pt}}m{#1}}

\usepackage{natbib,hyperref}
\bibliographystyle{chicago}  

\newcommand{\E}{\mathrm{E}}
\newcommand{\BLP}{\mathrm{BLP}}
\newcommand{\Var}{\mathrm{Var}}
\newcommand{\Cov}{\mathrm{Cov}}
\newcommand{\Corr}{\mathrm{Corr}}
\newcommand{\Prob}{\mathrm{P}}
\newcommand{\N}{\text{N}}

\topmargin=-1.5cm \textheight=23cm \oddsidemargin=-0.0cm
\evensidemargin=-0.0cm \textwidth=16.5cm
\newtheorem{ass}{Assumption}
\newtheorem{definit}{Definition}
\newtheorem{prop}{Proposition}
\newtheorem{thm}{Theorem}
\newtheorem{lem}{Lemma}
\newtheorem{conj}{Conjecture}
\newtheorem{cor}{Corollary}
\newtheorem{rem}{Remark}
\renewcommand{\thesubsection}{\arabic{section}.\arabic{subsection}}
\renewcommand{\thesubsubsection}{\arabic{section}.\arabic{subsection}.\arabic{subsubsection}}

\newcommand\independent{\protect\mathpalette{\protect\independenT}{\perp}}
\def\independenT#1#2{\mathrel{\rlap{$#1#2$}\mkern2mu{#1#2}}}

%Figure path
\def \figroot{stata/out/}
\def \tabroot{stata/out/}

\usepackage{epsfig,hyperref}

\hypersetup{
	pdftitle={ECMA31330 Final Project},    % title
	pdfauthor={Bronckers.Song.Zhang},     % author/Users/veronica/Documents/ECON21110/Final Project/ECON21110_FinalProject.tex
	pdfnewwindow=true,      % links in new window
	colorlinks=true,       % false: boxed links; true: colored links
	linkcolor=blue,          % color of internal links
	citecolor=red,        % color of links to bibliography
	filecolor=black,      % color of file links
	urlcolor=blue           % color of external links
}

\allowdisplaybreaks


\begin{document}
\begin{titlepage}
    \begin{center}
        \vspace*{1cm}
        \LARGE
        \textbf{PAPER TITLE\_Estimating Reaction to Welfare\\ Based on Question Wording \\}
        \vspace{0.5cm}
        \Large
        ECMA 31330 Final Project \\ 
        \vspace{0.8cm}
        \large
        Spring 2021
        \vfill
        \vspace{5cm}
        \textbf{Max Bronckers \\ Veronica Song \\ Dustin Zhang}
    \end{center}
\end{titlepage}


\section{Introduction} 
In this paper, we compare the use of Bayesian Additive Regression Trees (Green and Kern, 2012) versus Causal Forests (Wager and Athey, 2017) to estimate heterogenous treatment effects among survey responders using an empirical dataset. To measure the stigma around the word "welfare", we take a survey data of individual perception on public spending and find the impact of the question phrasing on the responses. In order to uncover the heterogenous effects of the question phrasing based on a suite of socioeconomic backgrounds of the respondent, Green and Kern utilizes Bayesian Additive Regression Trees (BART). BART has been selected as a popular method for modeling heterogenous treatment effects, 




\end{document}



















